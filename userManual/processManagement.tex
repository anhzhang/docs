
\section{Basic}

Introduce the basic ideas.

\lstset{language=C++, numbers=left, numberstyle=\tiny, stepnumber=1,
  numbersep=5pt, commentstyle=\scriptsize}
\begin{lstlisting}[caption={Basic Process Information},
                   basicstyle=\scriptsize,
                   label=listing:touch]
#include <fstream>
#include <iostream>
#include <mpi.h>

// mpic++ -o  mpiManagementExample mpiManagementExample.cpp 
// mpirun -np 4 --host localhost mpiManagementExample

int main(int argc,char **argv)
{
  int mpiResult;
  int numProcesses;
  int numtasks;
  int rank;
  char hostname[MPI_MAX_PROCESSOR_NAME];
  int len;

  mpiResult = MPI_Init (&argc,&argv);
  if(mpiResult!= MPI_SUCCESS)
    {
      std::cout << "MPI not started. Terminating the process." << std::en
        dl;
      MPI_Abort(MPI_COMM_WORLD,mpiResult);
    }

  MPI_Comm_size(MPI_COMM_WORLD,&numtasks);
  MPI_Comm_rank(MPI_COMM_WORLD,&rank);
  MPI_Get_processor_name(hostname, &len);
  std::cout << "Number of tasks= " <<  numtasks
            << " My rank= " << rank
            << " Running on " << hostname
            << std::endl;

  MPI_Finalize();
}
\end{lstlisting}

Basic commands include the following: \\
\begin{itemize}
\item MPI_Init
\item MPI_Comm_size
\item MPI_Comm_rank
\item MPI_Abort
\item MPI_Initialized
\end{itemize}


\section{Intermediate}

Intermediate ideas and commands associated include the following: \\
\begin{itemize}
\item MPI_Get_processor_name
\item MPI_Get_version
\item MPI_Initialized
\end{itemize}

Talk about groups and communicators.

\section{Advanced}

Intermediate ideas and commands associated include the following: \\
\begin{itemize}
\item MPI_Wtime
\item MPI_Wtick
\item Go into details about communicators?
\end{itemize}

Talk about groups and communicators.

%%% Local Variables: 
%%% mode: latex
%%% TeX-master: "OpenMPIUserManual"
%%% End: 
