\chapter{Process Management}

\section{Basic}

Introduce the basic ideas.

Basic commands include the following: \\
\begin{itemize}
\item \texttt{MPI\_Init}
\item \texttt{MPI\_Comm\_size}
\item \texttt{MPI\_Comm\_rank}
\item \texttt{MPI\_Abort}
\item \texttt{MPI\_Initialized}
\item \texttt{MPI\_Finalize}
\end{itemize}



\lstset{language=C++, numbers=left, numberstyle=\tiny, stepnumber=1,
  numbersep=5pt, commentstyle=\scriptsize}
\begin{lstlisting}[caption={Basic Process Information},
                   basicstyle=\scriptsize,
                   label=listing:basicProcess]
#include <fstream>
#include <iostream>
#include <mpi.h>

// mpic++ -o  mpiManagementExample mpiManagementExample.cpp 
// mpirun -np 4 --host localhost mpiManagementExample

int main(int argc,char **argv)
{
  // MPI job information.
  int  mpiResult;     // Used to check the results from MPI library calls
  int  numtasks;      // Total number of processes spawned for this job.
  int  rank;          // The unique number associated with this process.

  // Host Information 
  char hostname[MPI_MAX_PROCESSOR_NAME];
  int  len;

  // Initialize the session
  mpiResult = MPI_Init (&argc,&argv);
  if(mpiResult!= MPI_SUCCESS)
    {
      std::cout << "MPI not started. Terminating the process." << std::endl;
      MPI_Abort(MPI_COMM_WORLD,mpiResult);
    }

  // Get information about this session and this process 
  MPI_Comm_size(MPI_COMM_WORLD,&numtasks);  // get the number of processes
  MPI_Comm_rank(MPI_COMM_WORLD,&rank);      // get the rank of this process
  MPI_Get_processor_name(hostname, &len);   // Get the host name for
                                            // this process

  // Print out the information about this process.
  std::cout << "Number of tasks= " <<  numtasks
            << " My rank= " << rank
            << " Running on " << hostname
            << std::endl;

  // All done. Time to wrap it up.
  MPI_Finalize();
  return(0);
}
\end{lstlisting}


\section{Intermediate}

Intermediate ideas and commands associated include the following: \\
\begin{itemize}
\item \texttt{MPI\_Get\_processor\_name}
\item \texttt{MPI\_Get\_version}
\item \texttt{MPI\_Initialized}
\end{itemize}

Talk about groups and communicators.

Error handling and status.

\section{Advanced}

Intermediate ideas and commands associated include the following: \\
\begin{itemize}
\item \texttt{MPI\_Wtime}
\item \texttt{MPI\_Wtick}
\item Go into details about communicators?
\end{itemize}

Talk about groups and communicators.

Error handling and status.

%%% Local Variables: 
%%% mode: latex
%%% TeX-master: "OpenMPIUserManual"
%%% End: 
